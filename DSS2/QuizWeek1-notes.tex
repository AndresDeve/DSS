\documentclass[12pt]{article}
\usepackage{framed}
\begin{document}
	
%---------------------------------------------------------------------------------------------------------%

\subsection*{Question 1}
\Large
\texttt{R} was developed by statisticians working at
\begin{itemize}
	\item[(i)] Insightful
	\item[(ii)] Bell Labs
	\item[(iii)] The University of New South Wales
	\item[(iv)] The University of Auckland
\end{itemize}

%---------------------------------------------------------------------------------------------------------%
\newpage
\subsection*{Question 2}
\Large
The definition of free software consists of four freedoms (freedoms 0 through 3). \\ \\Which of the following is NOT one of the freedoms that are part of the definition?
\begin{itemize}
	\item[(i)] The freedom to study how the program works, and adapt it to your needs.
	\item[(ii)] The freedom to improve the program, and release your improvements to the public, so that the whole community benefits.
	\item[(iii)] The freedom to run the program, for any purpose.
	\item[(iv)] The freedom to sell the software for any price.
\end{itemize}

%---------------------------------------------------------------------------------------------------------%
\newpage
\subsection*{Question 3}
\Large
In \texttt{R} the following are all \textbf{atomic data types} EXCEPT
\begin{itemize}
	\item[(i)] \texttt{logical}
	\item[(ii)] \texttt{complex}
	\item[(iii)] \texttt{list}
	\item[(iv)] \texttt{integer}
\end{itemize}
\bigskip
\noindent \textbf{Remark}: Atomic would mean \textit{non divisible}, having no subsets.
%---------------------------------------------------------------------------------------------------------%
\newpage
\subsection*{Question 4}
\Large
If I execute the expression \texttt{x <- 4} in \texttt{R}
, what is the class of the object `\texttt{x}' as determined by the `\texttt{class()}' function?
\begin{itemize}
	\item[(i)] \texttt{vector}
	\item[(ii)] \texttt{numeric}
	\item[(iii)] \texttt{complex}
	\item[(iv)] \texttt{real}
\end{itemize}

\begin{framed}
\begin{verbatim}
x <- 4
class(x)

\end{verbatim}	
\end{framed}
\newpage
\subsection*{Built-In Data Sets} %Ready 1

Several data sets , intended as learning tools, are automatically installed when \texttt{R} is installed. Many more are installed within packages to complement learning to use those packages. One of these is the famous Iris data set, which is used in many data mining exercises.
\begin{itemize}
\item \texttt{iris}
\item \texttt{mtcars}
\item \texttt{Nile}
\end{itemize}
To see what data sets are available, simply type \texttt{data()}.
To load a data set, simply type in the name of the data set. Some data sets are very large. To just see the first few (or last)  rows, we use the \texttt{head()} function or alternatively the \texttt{tail()} function. The default number of rows of these commands is 6. Other numbers can be specified.
\begin{verbatim} 
> head(iris)
  Sepal.Length Sepal.Width Petal.Length Petal.Width Species
1          5.1         3.5          1.4         0.2  setosa
2          4.9         3.0          1.4         0.2  setosa
3          4.7         3.2          1.3         0.2  setosa
4          4.6         3.1          1.5         0.2  setosa
5          5.0         3.6          1.4         0.2  setosa
6          5.4         3.9          1.7         0.4  setosa
>
> tail(iris,4)
    Sepal.Length Sepal.Width Petal.Length Petal.Width   Species
147          6.3         2.5          5.0         1.9 virginica
148          6.5         3.0          5.2         2.0 virginica
149          6.2         3.4          5.4         2.3 virginica
150          5.9         3.0          5.1         1.8 virginic
\end{verbatim}
In many situations, it is useful to call a particular data set using the \texttt{attach()} command. This will save having to specify the data sets over repeated operations. The file can then be detached using 
 the \texttt{detach()} command.



%----------------------------------------------------------------%
\subsection*{Other Useful Commands}
\begin{framed}
\begin{verbatim}
mode(x)
str(x)
dim(x)
length(x)
\end{verbatim}	
\end{framed}

\noindent Try out these commands for some other objects, including inbuilt data sets \textbf{iris} and \textbf{Nile}.

\begin{framed}
\begin{verbatim}
iris   
Nile
Y <- "R"
Z <- c(TRUE,FALSE,TRUE)
\end{verbatim}
\end{framed}
%---------------------------------------------------------------------------------------------------------%
\newpage
\subsection*{Question 5}
\Large
What is the class of the object defined by the expression \texttt{x <- c(4, "a", TRUE)}?
\begin{itemize}
	\item[(i)] \texttt{mixed}
	\item[(i)] \texttt{logical}
	\item[(i)] \texttt{character}
	\item[(i)] \texttt{integer}
\end{itemize}

\begin{framed}
\begin{verbatim}
x <- c(4, "a", TRUE)
class(x)
\end{verbatim}	
\end{framed}
%---------------------------------------------------------------------------------------------------------%
\newpage
\subsection*{Question 6}
\Large
If I have two vectors \texttt{x <- c(1,3, 5)} and \texttt{y <- c(3, 2, 10)}, what is produced by the expression 
\texttt{rbind(x, y)}?

\begin{itemize}
	\item[(i)] a vector of length 2
	\item[(ii)] a 2 by 3 matrix
	\item[(iii)] a vector of length 3
	\item[(iv)] a 2 by 2 matrix
\end{itemize}

\begin{framed}
	\begin{verbatim}
	x <- c(1,3, 5)
	y <- c(3, 2, 10)
	rbind(x, y)
	\end{verbatim}	
\end{framed}
\begin{itemize}
\item Use the help file to find out what the commands \texttt{rbind()}, \texttt{cbind()} and \texttt{t()} do.
\item The convention is to specify the number of rows, then number of columns.
\item The tranpose operator ``\texttt{t()}" is actually really useful for creating reports from data contained in data frames.
\end{itemize}




%---------------------------------------------------------------------------------------------------------%
\newpage
\subsection*{Question 7}
\Large
A key property of vectors in \texttt{R} is that
\begin{itemize}
	\item[(i)] the length of a vector must be less than 32,768
	\item[(ii)] elements of a vector all must be of the same class
	\item[(iii)] elements of a vector can only be character or numeric
	\item[(iv)] elements of a vector can be of different classes
\end{itemize}




%---------------------------------------------------------------------------------------------------------%
\newpage
\subsection*{Question 8}
\Large
Suppose I have a list defined as \texttt{x <- list(2, "a", "b", TRUE)}. \\ \\ What does \texttt{x[[2]]} give me?\\ \\
\begin{itemize}
\item[(i)] a list containing the number 2 and the letter "a".
\item[(ii)] a character vector with the elements "a" and "b".
\item[(iii)] a character vector containing the letter "a".
\item[(iv)] a list containing a character vector with the elements "a" and "b".
\end{itemize}

\begin{framed}
	\begin{verbatim}
	x <- list(2, "a", "b", TRUE)
	x[[2]]
	\end{verbatim}	
\end{framed}

\begin{itemize}
	\item Try out \texttt{dim()} and \texttt{class()} on x also.
	\end{itemize}
%---------------------------------------------------------------------------------------------------------%
\newpage
\subsection*{Question 9}
\Large
Suppose I have a vector \texttt{x <- 1:4} and a vector \texttt{y <- 2}. \\ What is produced by the expression \texttt{x + y}?
\begin{itemize}
	\item[(i)] a numeric vector with elements 3, 4, 5, 6.
	\item[(ii)] an integer vector with elements 3, 2, 3, 4.
	\item[(iii)] a numeric vector with elements 3, 2, 3, 4.
	\item[(iv)] an integer vector with elements 3, 2, 3, 6.
\end{itemize}

%---------------------------------------------------------------------------------------------------------%
\newpage
\subsection*{Question 10}
\Large
Suppose I have a vector 
\begin{verbatim}
x <- c(3, 5, 1, 10, 12, 6) 
\end{verbatim}
and I want to set all elements of this vector that are less than 6 to be equal to zero. What \texttt{R} code achieves this?
\begin{itemize}
\item[(i)] \texttt{x[x > 0] <- 6}
\item[(ii)]\texttt{x[x < 6] <- 0}
\item[(iii)] \texttt{x[x > 6] <- 0}
\item[(iv)] \texttt{x[x == 6] <- 0}
\end{itemize}
\bigskip
\noindent This is called \textbf{Logical Indexing}. To get an idea of logical indexing, try out the following code snippets.
\begin{framed}
\begin{verbatim}
x <- c(3, 5, 1, 10, 12, 6) 
x>0
x<6
X==6
\end{verbatim}
\end{framed}
%---------------------------------------------------------------------------------------------------------%
\newpage
\subsection*{The airquality data set}

\textbf{Some Useful Commands}
As well as some of the commands we have seen earlier, it is worth getting to know the following commands also.
\begin{itemize}
\item[1] \texttt{head()} and \texttt{tail()}
\item[2] \texttt{names()}, \texttt{rownames()} and \texttt{colnames()}
\item[3] \texttt{summary()}
\item[4] \texttt{complete.cases()}
\item[5] \texttt{dim()}, \texttt{nrow()} and \texttt{ncol()}
\end{itemize}
Use the \texttt{help} command to find out what each of these commands do.
\begin{framed}
\begin{verbatim}
help(complete.cases)
\end{verbatim}
\end{framed}
\begin{framed}
	\begin{verbatim}
	> head(airquality)
	Ozone Solar.R  	Wind Temp Month Day
	1      41     190    7.4   67     5   1
	2      36     118    8.0   72     5   2
	3      12     149   12.6   74     5   3
	....
	\end{verbatim}
\end{framed}


\newpage
\subsection*{Inspecting the data set}
% Section 2

\begin{itemize}
\item[2a)] Dimensions
\item[2b)] Column names (i.e. variables names)
\item[2c)] structure of data frame
\end{itemize}
%Programming Assignment Question 3
Lets compute the dimensions of the data frame \textit{\textbf{iris}}, and also the length of the \textit{\textbf{Nile}} data set.
\begin{framed}
\begin{verbatim}
dim(iris)
nrow(iris)
ncol(iris)
length(Nile)
\end{verbatim}
\end{framed}

Data frames often have specifically named rows and columns. 
Lets find out the names of the variables (columns) and cases (rows) for the data sets \textit{\textbf{iris}} and \textit{\textbf{mtcars}}.

%Programming Assignment Question 1
\begin{framed}
\begin{verbatim}
names(iris)
colnames(iris)
rownames(iris) # simply the case numbers.

names(mtcars)
rownames(mtcars)
colnames(mtcars)
\end{verbatim}
\end{framed}

\newpage
\subsection*{The \texttt{summary()} command}
The \texttt{summary()} command can be used to extract a short statistical summary (if applicable) from each column of the data frame. 
If there are missing values, the frequency of missing values will also be listed for each column.

\begin{verbatim}
> summary(iris)
  Sepal.Length    Sepal.Width     Petal.Length    Petal.Width          Species  
 Min.   :4.300   Min.   :2.000   Min.   :1.000   Min.   :0.100   setosa    :50  
 1st Qu.:5.100   1st Qu.:2.800   1st Qu.:1.600   1st Qu.:0.300   versicolor:50  
 Median :5.800   Median :3.000   Median :4.350   Median :1.300   virginica :50  
 Mean   :5.843   Mean   :3.057   Mean   :3.758   Mean   :1.199                  
 3rd Qu.:6.400   3rd Qu.:3.300   3rd Qu.:5.100   3rd Qu.:1.800                  
 Max.   :7.900   Max.   :4.400   Max.   :6.900   Max.   :2.500                
\end{verbatim}


\newpage
\subsection*{Types of Data in Dataframes}
%Programming assignment Question 9
What is the data type of each column in the iris data set? To find out, we use the \texttt{str()} command.

\begin{framed}
\begin{verbatim}
str(iris)
\end{verbatim}
\end{framed}
The output of this command is given below. There are four numeric variables, and one factor (i.e. categorical) variable.
\begin{verbatim}
'data.frame':   150 obs. of  5 variables:
 $ Sepal.Length: num  5.1 4.9 4.7 4.6 5 5.4 4.6 5 4.4 4.9 ...
 $ Sepal.Width : num  3.5 3 3.2 3.1 3.6 3.9 3.4 3.4 2.9 3.1 ...
 $ Petal.Length: num  1.4 1.4 1.3 1.5 1.4 1.7 1.4 1.5 1.4 1.5 ...
 $ Petal.Width : num  0.2 0.2 0.2 0.2 0.2 0.4 0.3 0.2 0.2 0.1 ...
 $ Species     : Factor w/ 3 levels "setosa","versicolor",..: 1 1 1 1 1 1 1 1 1 1 ...
> 
\end{verbatim}


%--------------------------------------%

\subsection*{The \texttt{head()} and \texttt{tail()} function}
The head and tail functions can be used to access the first six and last six rows of a data frame. If a different number of rows is required, all you have to do is specify that number as an additional argument.
\begin{framed}
\begin{verbatim}
head(iris)     #First 6 rows
head(iris,2)   #First 2 rows

tail(iris)     #Last 6 rows
tail(iris,4)   #Last 4 rows
\end{verbatim}
\end{framed}

\newpage
%--------------------------------------%

\subsection*{Accessing a particular row or set of rows}

\begin{itemize}
\item Each value in a dataframe can be accessed directly by specifying the row and column i.e. \texttt{df[r,c]}.
\item To access a particular row, simply specify the row number, while leaving the column number blank i.e. \texttt{df[r,]}
\item To access a particular column, simple specify the column number, while leaving the row number blank i.e. \texttt{df[,c]}
\end{itemize}

\begin{framed}
\begin{verbatim}
iris[10,2]

iris[10,] 

Formaldehyde[,2]  
\end{verbatim}
\end{framed}

\begin{verbatim}
> iris[10,2]
[1] 3.1
> iris[10,] 
   Sepal.Length Sepal.Width Petal.Length Petal.Width Species
10          4.9         3.1          1.5         0.1  setosa
> 
> Formaldehyde[,2]  
[1] 0.086 0.269 0.446 0.538 0.626 0.782
\end{verbatim}
%--------------------------------------%
\subsection*{Missing data}
%Section 4
\begin{itemize}
\item Determining the number of missing data items
\item Performing statistical operations removing missing data
\end{itemize}

As stated previously, the \texttt{summary()} command can be used to determine the number of missing data items in a data frame. The additional argument \texttt{na.rm=TRUE} can also be used with certain functions (see the help file)

\begin{verbatim}
> X <- c(4,6,3,12,NA,8)
> mean(X)
[1] NA
>
> summary(X)
   Min. 1st Qu.  Median    Mean 3rd Qu.    Max.    NA's 
    3.0     4.0     6.0     6.6     8.0    12.0       1
>
> mean(X ,na.rm = TRUE)
[1] 6.6
\end{verbatim}

\newpage
\subsection*{Subsetting Data}
\subsection*{Logical and Relational Operator}

\begin{itemize}
\item AND - The logical operator is $\&$
\item OR - The logical operator is $\|$
\end{itemize}

\subsection*{Selection using the \texttt{subset()} Function}
The \texttt{subset( )} function is the easiest way to select variables and observation. 


\noindent \textbf{Example 1} \\
In the following example we will use the \textit{\textbf{iris}} data set, we select all rows that have a value of sepal length of 6 or more, and determine how many observations there are, and then compute the mean of the petal lengths. (The answer is 5.263, from the summary output).

\begin{framed}
\begin{verbatim}

#call the subset iris.2

iris.2 = subset(iris,iris$Sepal.Length >= 6)

dim(iris.2)

summary(iris.2)

\end{verbatim}
\end{framed}

\newpage
\noindent \textbf{Example 2} \\
There are three types of iris - setosa, versicoloir and virginica. Suppose we wish to compute the median of petal widths for the setosa irises only.

The equality operator is $==$.
\begin{verbatim}
> iris.setosa =subset(iris,iris$Species=="setosa")
> summary(iris.setosa)
  Sepal.Length    Sepal.Width     Petal.Length    Petal.Width   
 Min.   :4.300   Min.   :2.300   Min.   :1.000   Min.   :0.100  
 1st Qu.:4.800   1st Qu.:3.200   1st Qu.:1.400   1st Qu.:0.200  
 Median :5.000   Median :3.400   Median :1.500   Median :0.200  
 Mean   :5.006   Mean   :3.428   Mean   :1.462   Mean   :0.246  
 3rd Qu.:5.200   3rd Qu.:3.675   3rd Qu.:1.575   3rd Qu.:0.300  
 Max.   :5.800   Max.   :4.400   Max.   :1.900   Max.   :0.600  
       Species  
 setosa    :50  
 versicolor: 0  
 virginica : 0 
\end{verbatim} 
 
\noindent \textbf{Example 3} \\
In the following example we will use the \textit{\textbf{iris}} data set, we select all rows that have a value of sepal length of 6 or more, but have sepal width is at least 2.5.

\begin{verbatim}
> iris.3 = subset(iris,(iris$Sepal.Length >= 6)&(iris$Sepal.Width >=2.5))
> summary(iris.3)
  Sepal.Length    Sepal.Width     Petal.Length    Petal.Width   
 Min.   :6.000   Min.   :2.500   Min.   :4.000   Min.   :1.200  
 1st Qu.:6.300   1st Qu.:2.800   1st Qu.:4.750   1st Qu.:1.500  
 Median :6.500   Median :3.000   Median :5.300   Median :1.800  
 Mean   :6.641   Mean   :3.013   Mean   :5.313   Mean   :1.848  
 3rd Qu.:6.900   3rd Qu.:3.200   3rd Qu.:5.750   3rd Qu.:2.150  
 Max.   :7.900   Max.   :3.800   Max.   :6.900   Max.   :2.500  
       Species  
 setosa    : 0  
 versicolor:21  
 virginica :42  
\end{verbatim}

\newpage
\subsection*{Question 11}
\Large
In the dataset provided for this Quiz, what are the column names of the dataset?
\begin{itemize} 
\item[(i)] Ozone, Solar.R, Wind, Temp, Month, Day
\item[(ii)] Ozone, Solar.R, Wind
\item[(iii)] Month, Day, Temp, Wind
\item[(iv)] 1, 2, 3, 4, 5, 6
\end{itemize}
%---------------------------------------------------------------------------------------------------------%

\newpage
\subsection*{Sequences and Numerical Indexing}
\Large

A sequence of integers can be created usingn the colon symbol. The sequence may either be \textit{count-up} or \textit{count-down}.
\\
Importantly, the sequence will return all integers betwen the upper and lower bound, and including both the upper and lower bound (\textit{Other languages can be different in this respect})
\begin{framed}
	\begin{verbatim}
	0:5
	1:6
	-4:5
	10:1
	\end{verbatim}
\end{framed}

\noindent A contiguous group of rows from a a data frame may be extracted using the appropriate sequence of values as indices. Likewise for a contiguous group of columns

\begin{framed}
	\begin{verbatim}
	iris[1:6,]     # First Six Rows
	iris[,3:4]     # Third and Fourth Columns
	iris[1:40,2:3] # 40 Rows, Columns 2 and 3
	\end{verbatim}
\end{framed}

\section{Indexing and Subsetting }
\subsection{Relational and Logical Operators}

Relational operators allow for the comparison of values in vectors.
\begin{center}
\begin{tabular}{|c|c|}
  \hline
greater than &	$>$\\
less than&	$<$\\
equal to	&$==$\\
less than or equal to&	$<=$\\
greater than or equal to&	$>=$\\
not equal to	&$!=$\\
  \hline
\end{tabular}
\end{center}


\begin{itemize}
\item Note the difference of the equality operator "==" with assignment operator "=".

\item \& and \&\& indicate logical AND and $\|$ and $\|\|$ indicate logical OR.
\item The shorter form performs element-wise comparisons in much the same way as arithmetic operators. The longer form is appropriate for programming control-flow and typically preferred in "if" clauses.
\item We can use relational operators to subset vectors (as well as more complex data objects such as data frames, which we will meet later).
\item We specify the  relational condition in square brackets.
\item We can construct compound relational conditions too, using logical operators
\end{itemize}
\begin{framed}
\begin{verbatim}
> vec=1:19
> vec[vec<5]
[1] 1 2 3 4
> vec[(vec<6)|(vec>16)]
[1]  1  2  3  4  5 17 18 19
\end{verbatim}
\end{framed}





% End of relational and Logical Operators
\newpage
\subsection*{Question 12}
\Large
Extract the first 2 rows of the data frame and print them to the console. What does the output look like?

\begin{framed}
	\begin{verbatim}
		Ozone Solar.R Wind Temp Month Day
	1    41     190  7.4   67     5   1
	2    36     118  8.0   72     5   2
	\end{verbatim}
\end{framed}
\begin{framed}
	\begin{verbatim} 
		Ozone Solar.R Wind Temp Month Day
	1     7      NA  6.9   74     5  11
	2    35     274 10.3   82     7  17
	\end{verbatim}
\end{framed}
\begin{framed}
	\begin{verbatim} 
		Ozone Solar.R Wind Temp Month Day
	1    18     224 13.8   67     9  17
	2    NA     258  9.7   81     7  22
\end{verbatim}
\end{framed}	\begin{framed}
	\begin{verbatim} 
	Ozone Solar.R Wind Temp Month Day
	1     9      24 10.9   71     9  14
	2    18     131  8.0   76     9  29
	\end{verbatim}
\end{framed}

%---------------------------------------------------------------------------------------------------------%
\newpage

\subsection*{Question 13}
\Large
How many observations (i.e. rows) are in this data frame?
\begin{itemize}
	\item 129
	\item 153
	\item 160
	\item 45
\end{itemize}

%---------------------------------------------------------------------------------------------------------%
\newpage
\subsection*{Question 14}
\Large
Extract the last 2 rows of the data frame and print them to the console. What does the output look like?

\begin{framed}
	\begin{verbatim}
Ozone Solar.R Wind Temp Month Day
152    11      44  9.7   62     5  20
153   108     223  8.0   85     7  25
	\end{verbatim}
\end{framed}
\begin{framed}
	\begin{verbatim}
Ozone Solar.R Wind Temp Month Day
152    34     307 12.0   66     5  17
153    13      27 10.3   76     9  18
	\end{verbatim}
\end{framed}
\begin{framed}
	\begin{verbatim}
Ozone Solar.R Wind Temp Month Day
152    18     131  8.0   76     9  29
153    20     223 11.5   68     9  30
	\end{verbatim}
\end{framed}
\begin{framed}
	\begin{verbatim}
Ozone Solar.R Wind Temp Month Day
152    31     244 10.9   78     8  19
153    29     127  9.7   82     6   7
	\end{verbatim}
\end{framed}

%---------------------------------------------------------------------------------------------------------%
\newpage
\subsection*{Question 15}
\Large
What is the value of Ozone in the 47th row?
\begin{itemize}
\item[(i)] 63
\item[(ii)] 34
\item[(iii)] 18
\item[(iv)] 21
\end{itemize}

%---------------------------------------------------------------------------------------------------------%
\newpage
\subsection*{Question 16}
\Large
How many missing values (\texttt{NA}s) are in the \textit{Ozone} column of this data frame?

\begin{itemize}
	\item[(i)] 37
	\item[(ii)] 78
	\item[(iii)] 43
	\item[(iv)] 9
\end{itemize}

\begin{framed}
\begin{verbatim}
names(airquality)
attach(airquality)
is.na(Ozone)

!is.na(Ozone)

which(is.na(Ozone))

length(which(is.na(Ozone)))
detach(airquality)
\end{verbatim}
\end{framed}
%---------------------------------------------------------------------------------------------------------%
\newpage
\subsection*{Question 17}
\Large
What is the mean of the Ozone column in this dataset? Exclude missing values (coded as NA) from this calculation.

\begin{itemize}
	\item[(i)] 42.1
	\item[(ii)] 18.0
	\item[(iii)] 53.2
	\item[(iv)] 31.5
\end{itemize}


%---------------------------------------------------------------------------------------------------------%

\subsection*{Question 18}
\Large
Extract the subset of rows of the data frame where Ozone values are above 31 and Temp values are above 90. 

What is the mean of Solar.R in this subset?

\begin{itemize}
	\item[(i)] 185.9
	\item[(ii)] 212.8
	\item[(iii)] 334.0
	\item[(iv)] 205.0
\end{itemize}
%---------------------------------------------------------------------------------------------------------%
\newpage
\subsection*{Question 19}
\Large
What is the mean of "Temp" when "Month" is equal to 6?

\begin{itemize}
	\item[(i)] 90.2
	\item[(ii)] 85.6
	\item[(iii)] 79.1
	\item[(iv)] 75.3
\end{itemize}
%---------------------------------------------------------------------------------------------------------%

\subsection*{Question 20}
\Large
What was the maximum ozone value in the month of May (i.e. Month = 5)?


\begin{itemize}
	\item[(i)] 100
	\item[(ii)] 115
	\item[(iii)] 18
	\item[(iv)] 97
\end{itemize}
%---------------------------------------------------------------------------------------------------------%

\end{document}
