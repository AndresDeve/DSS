

\documentclass[11pt]{article} % use larger type; default would be 10pt
\usepackage{framed}
\usepackage[utf8]{inputenc} % set input encoding (not needed with XeLaTeX)
\usepackage{geometry} % to change the page dimensions
\geometry{a4paper} % or letterpaper (US) or a5paper or....


\usepackage{graphicx} % support the \includegraphics command and options

% \usepackage[parfill]{parskip} % Activate to begin paragraphs with an empty line rather than an indent

%%% PACKAGES
\usepackage{booktabs} % for much better looking tables
\usepackage{array} % for better arrays (eg matrices) in maths
\usepackage{paralist} % very flexible & customisable lists (eg. enumerate/itemize, etc.)
\usepackage{verbatim} % adds environment for commenting out blocks of text & for better verbatim
\usepackage{subfig} % make it possible to include more than one captioned figure/table in a single float
% These packages are all incorporated in the memoir class to one degree or another...
\usepackage{framed}

%%% HEADERS & FOOTERS
\usepackage{fancyhdr} % This should be set AFTER setting up the page geometry
\pagestyle{fancy} % options: empty , plain , fancy
\renewcommand{\headrulewidth}{0pt} % customise the layout...
\lhead{}\chead{}\rhead{}
\lfoot{}\cfoot{\thepage}\rfoot{}

%%% SECTION TITLE APPEARANCE
\usepackage{sectsty}
\allsectionsfont{\sffamily\mdseries\upshape} % (See the fntguide.pdf for font help)
% (This matches ConTeXt defaults)

%%% ToC (table of contents) APPEARANCE
\usepackage[nottoc,notlof,notlot]{tocbibind} % Put the bibliography in the ToC
\usepackage[titles,subfigure]{tocloft} % Alter the style of the Table of Contents
\renewcommand{\cftsecfont}{\rmfamily\mdseries\upshape}
\renewcommand{\cftsecpagefont}{\rmfamily\mdseries\upshape} % No bold!

\begin{document}
\tableofcontents
\newpage
%------------------------------------------------------%
\section{R programming Week 4}

\subsection*{The \texttt{grep()} function }
 What does the \texttt{grep()} function do when called with its default arguments?
 
 It returns the indices for strings in a character that match a given regular expression.

\texttt{grep(value = TRUE)} returns a character vector containing the selected elements of x (after coercion, preserving names but no other attributes).
\subsection*{The \texttt{grepl()} function }


\texttt{grepl} returns a logical vector (match or not for each element of x).
\subsection*{Regular Expressions}
%Question 7
\begin{framed}
\begin{verbatim}
^s(.*?)r
\end{verbatim}
\end{framed}
%------------------------------------------------------%
\subsection*{Generic Functions}
%Question 8
In the R system of classes and methods, what is a generic function?


\subsection*{S4 Methods}
What function is used to obtain the function body for an S4 method function?

Not \texttt{getS3method()}. 

What does the setOldClass function do?

\newpage
%----------------------- %
\subsection*{Question 1}
What is produced at the end of this snippet of R code?
\begin{verbatim}
set.seed(1)
rpois(5, 2)
\end{verbatim}
\begin{itemize}
\item[(i)] It is impossible to tell because the result is random
\item[(ii)] A vector with the numbers 1, 4, 1, 1, 5
\item[(iii)] A vector with the numbers 3.3, 2.5, 0.5, 1.1, 1.7
\item[(iv)] A vector with the numbers 1, 1, 2, 4, 1
\end{itemize}
%================================================ %
\newpage
\subsection*{Question 2}
What \texttt{R} function can be used to generate standard Normal random variables?

\begin{itemize}
\item[(i)] \texttt{dnorm}
\item[(ii)] \texttt{qnorm}
\item[(iii)] \texttt{rnorm}
\item[(iv)] \texttt{pnorm}
\end{itemize}
%================================================ %
\newpage
\subsection*{Question 3}
When simulating data, why is using the \texttt{set.seed()} function important?
\begin{itemize}
\item[(i)] It ensures that the sequence of random numbers starts in a specific place and is therefore reproducible.
\item[(ii)] It can be used to generate non-uniform random numbers.
\item[(iii)] It ensures that the sequence of random numbers is truly random.
\item[(iv)] It ensures that the random numbers generated are within specified boundaries.
\end{itemize}
%================================================ %
\newpage
\subsection*{Question 4}
Which function can be used to evaluate the inverse cumulative distribution function for the Poisson distribution?

\begin{itemize}
\item[(i)] \texttt{rpois}
\item[(ii)] \texttt{ppois}
\item[(iii)] \texttt{qpois}
\item[(iv)] \texttt{dpois}
\end{itemize}
%================================================ %
\newpage
\subsection*{Question 5}
What does the following code do?
\begin{framed}
\begin{verbatim}
set.seed(10)
x <- rbinom(10, 10, 0.5)
e <- rnorm(10, 0, 20)
y <- 0.5 + 2 * x + e
\end{verbatim}
\end{framed}
\begin{itemize}
\item[(i)] Generate uniformly distributed random data
\item[(ii)] Generate data from a Normal linear model
\item[(iii)] Generate random exponentially distributed data
\item[(iv)] Generate data from a Poisson generalized linear model
\end{itemize}
%========================================================= %
\newpage
\subsection*{Question 6}
What \texttt{R} function can be used to generate Binomial random variables?

\begin{itemize}
\item[(i)] \texttt{dbinom}
\item[(ii)] \texttt{qbinom}
\item[(iii)] \texttt{rbinom}
\item[(iv)] \texttt{pbinom}
\end{itemize}
%======================================== %
\newpage
\subsection*{Question 7}
What aspect of the \texttt{R} runtime does the profiler keep track of when an R expression is evaluated?

\begin{itemize}
\item[(i)] the function call stack
\item[(ii)] the global environment
\item[(iii)] the working directory
\item[(iv)] the package search list
\end{itemize}
%======================================== %
\newpage
\subsection*{Question 8}
Consider the following \texttt{R} code
\begin{framed}
\begin{verbatim}
library(datasets)
Rprof()
fit <- lm(y ~ x1 + x2)
Rprof(NULL)
\end{verbatim}
\end{framed}
(Assume that y, x1, and x2 are present in the workspace.) Without running the code, what percentage of the run time is spent in the '\texttt{lm}' function, based on the 'by.total' method of normalization shown in '\texttt{summaryRprof()}'?

\begin{itemize}
\item[(i)] 100\%
\item[(ii)] 23\%
\item[(iii)] 50\%
\item[(iv)] It is not possible to tell
\end{itemize}
%==========================================
\newpage
\subsection*{Question 9}
When using '\texttt{system.time()}', what is the user time?

\begin{itemize}
\item[(i)] It is a measure of network latency
\item[(ii)] It is the time spent by the CPU waiting for other tasks to finish
\item[(iii)] It is the "wall-clock" time it takes to evaluate an expression
\item[(iv)] It is the time spent by the CPU evaluating an expression
\end{itemize}
%========================================= %
\subsection*{Question 10}
If a computer has more than one available processor and R is able to take advantage of that, then which of the following is true when using '\texttt{system.time()}'?
\begin{itemize}
\item[(i)] elapsed time is 0
\item[(ii)] user time is always smaller than elapsed time
\item[(iii)] user time is 0
\item[(iv)] elapsed time may be smaller than user time
\end{itemize}
\end{document}