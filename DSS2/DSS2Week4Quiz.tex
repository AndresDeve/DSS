
\documentclass[11pt]{article} % use larger type; default would be 10pt
\usepackage{framed}
\usepackage[utf8]{inputenc} % set input encoding (not needed with XeLaTeX)
\usepackage{geometry} % to change the page dimensions
\geometry{a4paper} % or letterpaper (US) or a5paper or....


\usepackage{graphicx} % support the \includegraphics command and options

% \usepackage[parfill]{parskip} % Activate to begin paragraphs with an empty line rather than an indent

%%% PACKAGES
\usepackage{booktabs} % for much better looking tables
\usepackage{array} % for better arrays (eg matrices) in maths
\usepackage{paralist} % very flexible & customisable lists (eg. enumerate/itemize, etc.)
\usepackage{verbatim} % adds environment for commenting out blocks of text & for better verbatim
\usepackage{subfig} % make it possible to include more than one captioned figure/table in a single float
% These packages are all incorporated in the memoir class to one degree or another...
\usepackage{framed}

%%% HEADERS & FOOTERS
\usepackage{fancyhdr} % This should be set AFTER setting up the page geometry
\pagestyle{fancy} % options: empty , plain , fancy
\renewcommand{\headrulewidth}{0pt} % customise the layout...
\lhead{}\chead{}\rhead{}
\lfoot{}\cfoot{\thepage}\rfoot{}

%%% SECTION TITLE APPEARANCE
\usepackage{sectsty}
\allsectionsfont{\sffamily\mdseries\upshape} % (See the fntguide.pdf for font help)
% (This matches ConTeXt defaults)

%%% ToC (table of contents) APPEARANCE
\usepackage[nottoc,notlof,notlot]{tocbibind} % Put the bibliography in the ToC
\usepackage[titles,subfigure]{tocloft} % Alter the style of the Table of Contents
\renewcommand{\cftsecfont}{\rmfamily\mdseries\upshape}
\renewcommand{\cftsecpagefont}{\rmfamily\mdseries\upshape} % No bold!

\begin{document}
\tableofcontents
\newpage
%------------------------------------------------------%
\section{Computing For Data Analysis Week 4}

\subsection{The \texttt{grep()} function }
 What does the \texttt{grep()} function do when called with its default arguments?
 
 It returns the indices for strings in a character that match a given regular expression.

\subsection{Regular Expressions}
%Question 7
\begin{framed}
\begin{verbatim}
^s(.*?)r
\end{verbatim}
\end{framed}
%------------------------------------------------------%
\subsection{Generic Functions}
%Question 8
In the R system of classes and methods, what is a generic function?


\subsection{S4 Methods}
What function is used to obtain the function body for an S4 method function?

Not \texttt{getS3method()}. 

What does the setOldClass function do?

\end{document}
Week 4 Quiz Help

The due date for this quiz is Sun 28 Sep 2014 4:30 PM PDT.

In accordance with the Coursera Honor Code, I (Kevin O'Brien) certify that the answers here are my own work.
Question 1
What is produced at the end of this snippet of R code?
set.seed(1)
rpois(5, 2)
It is impossible to tell because the result is random
A vector with the numbers 1, 4, 1, 1, 5
A vector with the numbers 3.3, 2.5, 0.5, 1.1, 1.7
A vector with the numbers 1, 1, 2, 4, 1
Question 2
What R function can be used to generate standard Normal random variables?
dnorm
qnorm
rnorm
pnorm
Question 3
When simulating data, why is using the set.seed() function important?
It ensures that the sequence of random numbers starts in a specific place and is therefore reproducible.
It can be used to generate non-uniform random numbers.
It ensures that the sequence of random numbers is truly random.
It ensures that the random numbers generated are within specified boundaries.
Question 4
Which function can be used to evaluate the inverse cumulative distribution function for the Poisson distribution?
rpois
ppois
qpois
dpois
Question 5
What does the following code do?
set.seed(10)
x <- rbinom(10, 10, 0.5)
e <- rnorm(10, 0, 20)
y <- 0.5 + 2 * x + e
Generate uniformly distributed random data
Generate data from a Normal linear model
Generate random exponentially distributed data
Generate data from a Poisson generalized linear model
Question 6
What R function can be used to generate Binomial random variables?
dbinom
qbinom
rbinom
pbinom
Question 7
What aspect of the R runtime does the profiler keep track of when an R expression is evaluated?
the function call stack
the global environment
the working directory
the package search list
Question 8
Consider the following R code
library(datasets)
Rprof()
fit <- lm(y ~ x1 + x2)
Rprof(NULL)
(Assume that y, x1, and x2 are present in the workspace.) Without running the code, what percentage of the run time is spent in the 'lm' function, based on the 'by.total' method of normalization shown in 'summaryRprof()'?
100%
23%
50%
It is not possible to tell
Question 9
When using 'system.time()', what is the user time?
It is a measure of network latency
It is the time spent by the CPU waiting for other tasks to finish
It is the "wall-clock" time it takes to evaluate an expression
It is the time spent by the CPU evaluating an expression
Question 10
If a computer has more than one available processor and R is able to take advantage of that, then which of the following is true when using 'system.time()'?
elapsed time is 0
user time is always smaller than elapsed time
user time is 0
elapsed time may be smaller than user time
