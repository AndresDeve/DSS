\documentclass[]{article}

\usepackage{framed}
%opening
\title{Coursera - Computing For Data Analysis Week 4}

%https://github.com/pcward/coursera_programming-r/blob/master/rankhospital.r
%https://github.com/ConAntonakos/Computing_for_Data_Analysis/blob/master/rankhospital.R

% http://rpubs.com/flintt/3694
\begin{document}

\maketitle



\section{Week 4 Programming Assignment Part 2}

\subsection{Ages of homicide victims}

The goal of this part is to write a function called agecount that returns the number of
homicide victims of a given age. For most (but not all) records there is an indication of the
age of the victim.

\bigskip
\noindent
Your function should take one argument, the age of the victim(s), extract
the age of the victim from each record and then return a count of the number of victims of
the specied age.
\begin{itemize}
\item The argument passed to agecount should be a positive integer, but you do not need
to check for this.
\item If a record does not contain any age information, the record should be ignored.
\item The function should return an integer indicating the number of victims of a given age.
\item Your function should read the homicides dataset in the manner indicated above.
\end{itemize}
\end{document}