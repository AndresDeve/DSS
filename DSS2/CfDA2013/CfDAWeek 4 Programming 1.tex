\documentclass[]{article}

\usepackage{framed}
%opening
\title{Coursera - Computing For Data Analysis Week 4}

%https://github.com/pcward/coursera_programming-r/blob/master/rankhospital.r
%https://github.com/ConAntonakos/Computing_for_Data_Analysis/blob/master/rankhospital.R

% http://rpubs.com/flintt/3694
\begin{document}

\maketitle



\section{Week 4 Programming Assignment Part 1}

\subsection{How many of each cause of homicide?}

The goal of this problem is to count the number of different types of homicides that are
in this dataset. In each record there is a field with the word ``Cause" in it indicating the
cause of death (e.g. ``Cause: shooting"). The basic goal is to extract this field and count the
number of instances of each cause.

\bigskip
\noindent
Write a function named count that takes one argument, a character string indicating the
cause of death. The function should then return an integer representing the number of
homicides from that cause in the dataset. If no cause of death is specified, then the function
should return an error message via the stop function.

\begin{itemize}
\item  Your function should read the homicides dataset in the manner indicated above.
\item The options for cause of death are ``asphyxiation", ``blunt force", ``other", ``shooting",
``stabbing", ``unknown". 
\item No other causes are allowed. If a cause of death is specified
that is not one of these, then the function should throw an error with the stop function.
\item Note that some homicides in the dataset do not have a cause of death listed and those
records should be ignored.
\item Your function should deal with some irregularities in the dataset like capitalization.
For example ``Shooting" and ``shooting" should be counted as the same cause of death.
\end{itemize}
\end{document}