swirl Programming Assignment: Instructions Help 



swirl is a software package that turns the R console into an interactive learning environment. In this programming assignment, you'll have the opportunity to earn up to 5 extra credit points while using swirl to practice some key concepts from this course.
 
0. First things first
 •
You must have the most recent version of swirl to complete this assignment.
 
•
swirl requires R 3.0.2 or later. If you have an older version of R, please update before going any further. If you're not sure what version of R you have, type R.version.string at the R prompt.
 
•
If you are on a Linux operating system, please visit our Installing swirl on Linux page for special instructions.
 

1. Install swirl
 
Since swirl is an R package, you can easily install it by entering a single command from the R console:
 install.packages("swirl") 
2. Load and run swirl
 
Every time you want to use swirl, just load the package and start the program. From the R console:
 library(swirl)
swirl() 
3. Install the R Programming course
 
swirl offers a variety of interactive courses, but for our purposes, you want the one called R Programming. If this is your first time using swirl, it will prompt you to install the R Programming course automatically. If you've used swirl in the past, you will need to type the following from the R prompt:
 install_from_swirl("R Programming") 
4. Complete the lessons
 
There are 11 lessons in the R Programming course covering a variety of important topics.
 
Each completed lesson is worth one extra credit point. However, the maximum number of points you may earn for the assignment is capped at 5. Regardless, these lessons will give you valuable practice and you are encouraged to complete as many as possible. If you skip() more than one question in a lesson, you will not receive credit for that lesson.
 
5. Get extra credit for your work!
 
Upon completing each lesson, swirl will ask for your Coursera credentials:
 •Course ID: rprog-007
 •Submission login (email): The email address associated with your Coursera account
 •Submission password: This is NOT the password that you use to log into the Coursera website. Your submission password can be found at the top of the Programming Assignments page.
 
Once you've entered and confirmed this information, swirl will attempt to notify Coursera automatically. If something goes wrong with automatic submission, you'll have the option to retry or submit manually.
 
If you need help...
 •
Visit the Frequently Asked Questions (FAQ) page to see if you can answer your own question immediately.
 
•
Search the swirl Programming Assignment sub-forum, which is located on the Discussion Forums page for this course.
 
•
If you still can't find an answer to your question, then create a new thread under the swirl Programming Assignment sub-forum and provide the following information:
 ◦A descriptive title
 ◦Any input/output from the console (copy & paste) or a screenshot
 ◦The output from sessionInfo()
 

Good luck and have fun!
 
For more information on swirl, visit swirlstats.com.
