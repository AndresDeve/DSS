
% CFDA
%---------------------------------------------------------------------------------------------------------%
\begin{frame}
\frametitle{Question 1}
\Large
R was developed by statisticians working at
\begin{itemize}
\item Insightful
\item Bell Labs
\item The University of New South Wales
\item The University of Auckland
\end{itemize}
\end{frame}
%---------------------------------------------------------------------------------------------------------%
\begin{frame}
\frametitle{Question 2}
\Large
The definition of free software consists of four freedoms (freedoms 0 through 3). \\ Which of the following is NOT one of the freedoms that are part of the definition?
\begin{itemize}
\item The freedom to study how the program works, and adapt it to your needs.
\item The freedom to improve the program, and release your improvements to the public, so that the whole community benefits.
\item The freedom to run the program, for any purpose.
\item The freedom to sell the software for any price.
\end{itemize}
\end{frame}
%---------------------------------------------------------------------------------------------------------%
\begin{frame}
\frametitle{Question 3}
\Large
In \texttt{R} the following are all \textbf{atomic data types} EXCEPT
\begin{itemize}
\item \texttt{logical}
\item \texttt{complex}
\item \texttt{list}
\item \texttt{integer}
\end{frame}
%---------------------------------------------------------------------------------------------------------%
\begin{frame}
\frametitle{Question 4}
\Large
If I execute the expression x <- 4 in R, what is the class of the object `x' as determined by the `class()' function?
\begin{itemize}
\item \texttt{vector}
\item \texttt{numeric}
\item \texttt{complex}
\item \texttt{real}
\end{itemize}
\end{frame}
%---------------------------------------------------------------------------------------------------------%
\begin{frame}
\frametitle{Question 5}
\Large
What is the class of the object defined by the expression x <- c(4, "a", TRUE)?
mixed
logical
character
integer
\end{frame}
%---------------------------------------------------------------------------------------------------------%
\begin{frame}
\frametitle{Question 6}
\Large
If I have two vectors x <- c(1,3, 5) and y <- c(3, 2, 10), what is produced by the expression rbind(x, y)?
a vector of length 2
a 2 by 3 matrix
a vector of length 3
a 2 by 2 matrix
\end{frame}
%---------------------------------------------------------------------------------------------------------%
\begin{frame}
\frametitle{Question 7}
\Large
A key property of vectors in R is that
the length of a vector must be less than 32,768
elements of a vector all must be of the same class
elements of a vector can only be character or numeric
elements of a vector can be of different classes
\end{frame}
%---------------------------------------------------------------------------------------------------------%
\begin{frame}
\frametitle{Question 8}
\Large
Suppose I have a list defined as x <- list(2, "a", "b", TRUE). What does x[[2]] give me?
a list containing the number 2 and the letter "a".
a character vector with the elements "a" and "b".
a character vector containing the letter "a".
a list containing a character vector with the elements "a" and "b".
\end{frame}
%---------------------------------------------------------------------------------------------------------%
\begin{frame}
\frametitle{Question 9}
\Large
Suppose I have a vector $x <- 1:4$ and a vector y <- 2. What is produced by the expression x + y?
\begin{itemize}
\item a numeric vector with elements 3, 4, 5, 6.
\item an integer vector with elements 3, 2, 3, 4.
\item a numeric vector with elements 3, 2, 3, 4.
\item an integer vector with elements 3, 2, 3, 6.
\end{itemize}
\end{frame}
%---------------------------------------------------------------------------------------------------------%
\begin{frame}
\frametitle{Question 10}
\Large
Suppose I have a vector 
\begin{verbatim}
x <- c(3, 5, 1, 10, 12, 6) 
\end{verbatim}
and I want to set all elements of this vector that are less than 6 to be equal to zero. What R code achieves this?
x[x > 0] <- 6
x[x < 6] <- 0
x[x > 6] <- 0
x[x == 6] <- 0
\end{frame}
%---------------------------------------------------------------------------------------------------------%
\begin{frame}
\frametitle{Question 11}
\Large
In the dataset provided for this Quiz, what are the column names of the dataset?
Ozone, Solar.R, Wind, Temp, Month, Day
Ozone, Solar.R, Wind
Month, Day, Temp, Wind
1, 2, 3, 4, 5, 6
\end{frame}
%---------------------------------------------------------------------------------------------------------%
\begin{frame}
\frametitle{Question 12}
\Large
Extract the first 2 rows of the data frame and print them to the console. What does the output look like?

  Ozone Solar.R Wind Temp Month Day
1    41     190  7.4   67     5   1
2    36     118  8.0   72     5   2
 
  Ozone Solar.R Wind Temp Month Day
1     7      NA  6.9   74     5  11
2    35     274 10.3   82     7  17
 
  Ozone Solar.R Wind Temp Month Day
1    18     224 13.8   67     9  17
2    NA     258  9.7   81     7  22
 
  Ozone Solar.R Wind Temp Month Day
1     9      24 10.9   71     9  14
2    18     131  8.0   76     9  29
\end{frame}
%---------------------------------------------------------------------------------------------------------%
\begin{frame}
\frametitle{Question 13}
\Large
How many observations (i.e. rows) are in this data frame?
129
153
160
45
\end{frame}
%---------------------------------------------------------------------------------------------------------%
\begin{frame}
\frametitle{Question 14}
\Large
Extract the last 2 rows of the data frame and print them to the console. What does the output look like?

\begin{verbatim}
    Ozone Solar.R Wind Temp Month Day
152    11      44  9.7   62     5  20
153   108     223  8.0   85     7  25
 
    Ozone Solar.R Wind Temp Month Day
152    34     307 12.0   66     5  17
153    13      27 10.3   76     9  18
\end{verbatim}
\end{frame}
%---------------------------------------------------------------------------------------------------------%
\begin{frame}
\frametitle{Question 14 - Continued}
\Large
\begin{verbatim} 
    Ozone Solar.R Wind Temp Month Day
152    18     131  8.0   76     9  29
153    20     223 11.5   68     9  30
 
    Ozone Solar.R Wind Temp Month Day
152    31     244 10.9   78     8  19
153    29     127  9.7   82     6   7
\end{verbatim}
\end{frame}
%---------------------------------------------------------------------------------------------------------%
\begin{frame}
\frametitle{Question 15}
\Large
What is the value of Ozone in the 47th row?
63
34
18
21
\end{frame}
%---------------------------------------------------------------------------------------------------------%
\begin{frame}
\frametitle{Question 16}
\Large
How many missing values are in the Ozone column of this data frame?
37
78
43
9
\end{frame}
%---------------------------------------------------------------------------------------------------------%
\begin{frame}
\frametitle{Question 17}
\Large
What is the mean of the Ozone column in this dataset? Exclude missing values (coded as NA) from this calculation.
42.1
18.0
53.2
31.5
\end{frame}
%---------------------------------------------------------------------------------------------------------%
\begin{frame}
\frametitle{Question 18}
\Large
Extract the subset of rows of the data frame where Ozone values are above 31 and Temp values are above 90. What is the mean of Solar.R in this subset?
185.9
212.8
334.0
205.0
\end{frame}
%---------------------------------------------------------------------------------------------------------%
\begin{frame}
\frametitle{Question 19}
\Large
What is the mean of "Temp" when "Month" is equal to 6?
90.2
85.6
79.1
75.3
\end{frame}
%---------------------------------------------------------------------------------------------------------%
\begin{frame}
\frametitle{Question 20}
\Large
What was the maximum ozone value in the month of May (i.e. Month = 5)?
100
115
18
97
\end{frame}
%---------------------------------------------------------------------------------------------------------%

\end{document}
