\documentclass[french]{article}
\usepackage[utf8]{inputenc}
\usepackage{framed}
\usepackage[T1]{fontenc}
\usepackage{lmodern}
\usepackage[a4paper]{geometry}
\usepackage{babel}
\begin{document}
	\begin{itemize}
		\item A pair of functions that cache the inverse of a matrix
		
		
		\item this creates a special matrix object that can cache its inverse
	\end{itemize}
\begin{framed}
\begin{verbatim}

makeCacheMatrix <- function( m = matrix() ) {
	
	## Initialize the inverse property
	i <- NULL
	
	## Method to set the matrix
	set <- function( matrix ) {
		m <<- matrix
		i <<- NULL
	}
	
	## Method the get the matrix
	get <- function() {
		## Return the matrix
		m
	}
	
	## Method to set the inverse of the matrix
	setInverse <- function(inverse) {
		i <<- inverse
	}
	
	## Method to get the inverse of the matrix
	getInverse <- function() {
		## Return the inverse property
		i
	}
	
	## Return a list of the methods
	list(set = set, get = get,
	setInverse = setInverse,
	getInverse = getInverse)
}
\end{verbatim}
\end{framed}

\newpage
Compute the inverse of the special matrix returned by "makeCacheMatrix"
above. If the inverse has already been calculated (and the matrix has not
changed), then the "cachesolve" should retrieve the inverse from the cache.

\begin{framed}
\begin{verbatim}
cacheSolve <- function(x, ...) {

## Return a matrix that is the inverse of 'x'
m <- x$getInverse()

## Just return the inverse if its already set
if( !is.null(m) ) {
message("getting cached data")
return(m)
}

## Get the matrix from our object
data <- x$get()

## Calculate the inverse using matrix multiplication
m <- solve(data) %*% data

## Set the inverse to the object
x$setInverse(m)

## Return the matrix
m
}
\end{verbatim}
\end{framed}
\end{document}