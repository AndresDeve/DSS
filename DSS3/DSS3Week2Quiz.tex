\documentclass[12pt]{article}
\usepackage{framed}
\usepackage{graphicx}
\begin{document}
%---------------------------------------------------------------------------------------------%
\newpage
\subsection*{Week 2} 

Week 2 of Getting and Cleaning Data: Extracting Data From Databases and the Web
Welcome to Week 2 of Getting and Cleaning Data! The primary goal is to introduce you to the most common data storage systems and the appropriate tools to extract data from web or from databases like MySQL. 

Remember that the Course Project is open and ongoing. It is due BEFORE 11:30 PM UTC on the Sunday at the end of Week 3, but please don't put it off until the last minute. To access Course Project instructions and submission interface, click the Course Project link in the left navigation bar.  

With the skills you learn this week you should be able to start on the basic data extraction that will form the beginnings of your project. 

\newpage
\subsection*{Question 1}
\textbf{Question}
\begin{itemize}
	\item Register an application with the Github API here 
	\begin{verbatim}
	https://github.com/settings/applications. 
	\end{verbatim}
	\item Access the API to get information on your instructors repositories 
	\item (hint: this is the url you want "https://api.github.com/users/jtleek/repos"). \item Use this data to find the time that the datasharing repo was created. \item What time was it created?
	\item This tutorial may be useful
	\begin{verbatim} (https://github.com/hadley/httr/blob/master/demo/oauth2-github.r).
	\end{verbatim}
	\item You may also need to run the code in the base R package and not R studio.
\end{itemize}

\noindent \textbf{Options} \\
\begin{itemize}
\item[(i)] 2012-06-20T18:39:06Z
\item[(ii)] 2014-03-05T16:11:46Z
\item[(iii)] 2014-01-04T21:06:44Z
\item[(iv)] 2013-11-07T13:25:07Z
\end{itemize}

\newpage
\subsection*{The sqldf package}
%---------------------------------------------------------------------------------------------%
\newpage
\subsection*{Question 2}
The \textbf{sqldf} package allows for execution of SQL commands on \texttt{R} data frames. We will use the \textbf{sqldf} package to practice the queries we might send with the \texttt{dbSendQuery} command in RMySQL. 

\bigskip
\noindent Download the American Community Survey data and load it into an R object called
\texttt{ acs}.

\begin{verbatim}
https://d396qusza40orc.cloudfront.net/getdata%2Fdata%2Fss06pid.csv 
\end{verbatim}

\noindent Which of the following commands will select only the data for the probability weights \texttt{pwgtp1} with ages less than 50?
\begin{itemize}
\item[(i)] \texttt{sqldf("select * from acs where AGEP < 50 and pwgtp1")}
\item[(ii)] \texttt{sqldf("select * from acs")}
\item[(iii)] \texttt{sqldf("select * from acs where AGEP < 50")}
\item[(iv)] \texttt{sqldf("select pwgtp1 from acs where AGEP < 50")}
\end{itemize}

\begin{framed}
\begin{verbatim}
names(acs)
\end{verbatim}
\end{framed}
%---------------------------------------------------------------------------------------------%
\newpage
\subsection*{Question 3}
Using the same data frame you created in the previous problem, what is the equivalent function to \texttt{unique(acs\$AGEP)}
\begin{itemize}
\item[(i)] \texttt{sqldf("select unique AGEP from acs")}
\item[(ii)] \texttt{sqldf("select distinct pwgtp1 from acs")}
\item[(iii)] \texttt{sqldf("select AGEP where unique from acs")}
\item[(iv)] \texttt{sqldf("select distinct AGEP from acs")}
\end{itemize}

%---------------------------------------------------------------------------------------------%
\newpage
\subsection*{Question 4}
How many characters are in the 10th, 20th, 30th and 100th lines of HTML from this page: 
\begin{verbatim}
http://biostat.jhsph.edu/~jleek/contact.html 
\end{verbatim}
(Hint: the \texttt{nchar()} function in \texttt{R} may be helpful)
\begin{itemize}
\item[(i)] 43 99 8 6
\item[(ii)] 45 31 7 31
\item[(iii)] 43 99 7 25
\item[(iv)] 45 31 7 25
\item[(v)] 45 0 2 2
\item[(vi)] 45 31 2 25
\item[(vii)] 45 92 7 2
\end{itemize}

%---------------------------------------------------------------------------------------------%
\newpage
\subsection*{Question 5}
Read this data set into R and report the sum of the numbers in the fourth column. 
\begin{verbatim}
https://d396qusza40orc.cloudfront.net/getdata%2Fwksst8110.for 
\end{verbatim} 

\noindent Original source of the data:
\begin{verbatim}
http://www.cpc.ncep.noaa.gov/data/indices/wksst8110.for 
\end{verbatim} 

\textit{(Hint this is a fixed width file (fwf) format)}
\begin{itemize}
\item 32426.7
\item 35824.9
\item 222243.1
\item 36.5
\item 28893.3
\item 101.83
\end{itemize}

\begin{framed}
\begin{verbatim}
help(read.fwf)
\end{verbatim}	
\end{framed}
\end{document}

%-----------------------------------------------------------------------------------------------------%
uiz 2
Attempts	Score
1/3	15/15
Question 1

Register an application with the Github API here https://github.com/settings/applications. Access the API to get information on your instructors repositories (hint: this is the url you want "https://api.github.com/users/jtleek/repos"). Use this data to find the time that the datasharing repo was created. What time was it created? This tutorial may be useful (https://github.com/hadley/httr/blob/master/demo/oauth2-github.r). You may also need to run the code in the base R package and not R studio.

Answer

2013-11-07T13:25:07Z

Explanation

> library(httr)
> oauth_endpoints("github")
> myapp <- oauth_app("github", "ClientID", "ClientSecret")
> github_token <- oauth2.0_token(oauth_endpoints("github"), myapp)
> req <- GET("https://api.github.com/rate_limit", config(token = github_token))
> stop_for_status(req)
> content(req)
> BROWSE("https://api.github.com/users/jtleek/repos",authenticate("Access Token","x-oauth-basic","basic"))
Question 2

The sqldf package allows for execution of SQL commands on R data frames. We will use the sqldf package to practice the queries we might send with the dbSendQuery command in RMySQL. Download the American Community Survey data and load it into an R object called

acs
https://d396qusza40orc.cloudfront.net/getdata%2Fdata%2Fss06pid.csv

Which of the following commands will select only the data for the probability weights pwgtp1 with ages less than 50?

Answer

sqldf("select pwgtp1 from acs where AGEP < 50")

Explanation

> library(sqldf)
> acs <- read.csv("./getdata-data-ss06pid.csv", header=T, sep=",")
> sqldf("select pwgtp1 from acs where AGEP < 50")
Question 3

Using the same data frame you created in the previous problem, what is the equivalent function to unique(acs$AGEP)

Answer

sqldf("select distinct AGEP from acs")

Explanation

> sqldf("select distinct AGEP from acs")
Question 4

How many characters are in the 10th, 20th, 30th and 100th lines of HTML from this page:

http://biostat.jhsph.edu/~jleek/contact.html

(Hint: the nchar() function in R may be helpful)

Answer

45 31 7 25

Explanation

> hurl <- "http://biostat.jhsph.edu/~jleek/contact.html" 
> con <- url(hurl)
> htmlCode <- readLines(con)
> close(con)
> sapply(htmlCode[c(10, 20, 30, 100)], nchar)
<meta name="Distribution" content="Global" />               <script type="text/javascript"> 
                                           45                                            31 
                                        })();                 \t\t\t\t<ul class="sidemenu"> 
                                            7                                            25 
Question 5

Read this data set into R and report the sum of the numbers in the fourth column.

https://d396qusza40orc.cloudfront.net/getdata%2Fwksst8110.for

Original source of the data: http://www.cpc.ncep.noaa.gov/data/indices/wksst8110.for

(Hint this is a fixed width file format)

Answer

32426.7

Explanation

> data <- read.csv("./getdata-wksst8110.for", header = TRUE)
> file_name <- "./getdata-wksst8110.for"
> df <- read.fwf(file=file_name,widths=c(-1,9,-5,4,4,-5,4,4,-5,4,4,-5,4,4), skip=4)
> sum(df[, 4])
> [1] 32426.7
