1. Install swirl
Since swirl is an R package, you can easily install it by entering a single command from the R console:

install.packages("swirl")
2. Load swirl and install the Getting and Cleaning Data course
From the R console:

library(swirl)
install_from_swirl("Getting and Cleaning Data")
swirl()
3. Complete the lessons
There are 4 lessons in the Getting and Cleaning Data course covering a variety of important topics.

Each completed lesson is worth one extra credit point. However, the maximum number of points you may earn for the assignment is capped at 3. Regardless, these lessons will give you valuable practice and you are encouraged to complete as many as possible. If you skip() more than one question in a lesson, you will not receive credit for that lesson.

4. Get extra credit for your work!
Upon completing each lesson, swirl will ask for your Coursera credentials:

Course ID: getdata-007
Submission login (email): The email address associated with your Coursera account
Submission password: This is NOT the password that you use to log into the Coursera website. Your submission password can be found at the top of the Programming Assignments page.
Once you've entered and confirmed this information, swirl will attempt to notify Coursera automatically. If something goes wrong with automatic submission, you'll have the option to retry or submit manually.

If you need help...
Visit the Frequently Asked Questions (FAQ) page to see if you can answer your own question immediately.

Search the swirl Programming Assignment sub-forum, which is located on the Discussion Forums page for this course.

If you still can't find an answer to your question, then create a new thread under the swirl Programming Assignment sub-forum and provide the following information:

A descriptive title
Any input/output from the console (copy & paste) or a screenshot
The output from sessionInfo()
Good luck and have fun!
For more information on swirl, visit swirlstats.com.
