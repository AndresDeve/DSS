\documentclass[]{article}

\usepackage{framed}
\usepackage{graphicx}
\usepackage{amsmath}
\usepackage{amssymb}

\voffset=-1.5cm
\oddsidemargin=0.0cm
\textwidth = 470pt
\begin{document}


\section{Week 3 Quiz}

\subsection*{\texttt{file.path()}}

This command will construct the path to a file from components in a platform-independent way. 

Suppose there is a file called "RLogo.jpg" in the working directory.
\begin{framed}
\begin{verbatim}

file <- file.path(getwd(), "RLogo.jpg")

\end{verbatim}
\end{framed}
%-----------------------------------------------------------------------------------%
\subsection*{\texttt{download.file()}}

This function can be used to download a file from the Internet. 

\begin{description}
\item[\texttt{mode}] The mode with which to write the file. Useful values are ``w" (text), ``wb" (binary), ``a" (append) and ``ab".
\end{description}
%-----------------------------------------------------------------------------------%

\subsection*{\texttt{readJPEG()}}

This command , from the \texttt{R} package \textbf{\textit{jpeg}}, will read a bitmap image stored in the JPEG format 
\begin{framed}
\begin{verbatim}
# read a sample file (R logo)
img <- readJPEG(system.file("img", "Rlogo.jpg", package="jpeg"))
 
# read it also in native format
img.n <- readJPEG(system.file("img", "Rlogo.jpg", package="jpeg"), TRUE)
 
# if your R supports it, we'll plot it
if (exists("rasterImage")) { # can plot only in R 2.11.0 and higher
  plot(1:2, type='n')
 
  rasterImage(img, 1.2, 1.27, 1.8, 1.73)
  rasterImage(img.n, 1.5, 1.5, 1.9, 1.8)
}
\end{verbatim}
\end{framed}

\begin{verbatim}
http://www.inside-r.org/packages/cran/jpeg/docs/readJPEG
\end{verbatim}
%------------------------------------------------------------------------------------------------------------%
\subsection*{Question 1}
The American Community Survey distributes downloadable data about United States communities. Download the 2006 microdata survey about housing for the state of Idaho using \texttt{download.file()} from here: 
\begin{verbatim}
https://d396qusza40orc.cloudfront.net/getdata%2Fdata%2Fss06hid.csv 
\end{verbatim}
and load the data into \texttt{R}. The code book, describing the variable names is here: 
\begin{verbatim}
https://d396qusza40orc.cloudfront.net/getdata%2Fdata%2FPUMSDataDict06.pdf 
\end{verbatim}

\noindent Create a logical vector that identifies the households on greater than 10 acres who sold more than $\$10,000$ worth of agriculture products. 
\\
Assign that logical vector to the variable \texttt{agricultureLogical}. Apply the \texttt{which()} function like this to identify the rows of the data frame where the logical vector is TRUE. which(agricultureLogical) 
\\
What are the first 3 values that result?
\begin{itemize}
\item[(1)] 236, 238, 262
\item[(2)] 25, 36, 45
\item[(3)] 403, 756, 798
\item[(4)] 125, 238,262
\end{itemize}
\begin{framed}
\begin{verbatim}
> url <- "https://d396qusza40orc.cloudfront.net/getdata%2Fdata%2Fss06hid.csv"
> file <- file.path(getwd(), "ss06hid.csv")
> download.file(url, file, method = "curl")
> dt <- data.table(read.csv(file))
> agricultureLogical <- (dt$ACR == 3) & (dt$AGS == 6)
> which(agricultureLogical)[1:3]
[1] 125 238 262
\end{verbatim}
\end{framed}
\newpage
%=====================================================%
\subsection*{Question 2}
\textbf{Instructions}
\begin{itemize}
\item Using the \textbf{jpeg} package read in the following picture of your instructor into \texttt{R} 

\begin{verbatim}
https://d396qusza40orc.cloudfront.net/getdata%2Fjeff.jpg 
\end{verbatim}

\item Use the parameter \texttt{native=TRUE}. 
\item What are the 30th and 80th quantiles of the resulting data? 

\item \textit{(some Linux systems may produce an answer 638 different for the 30th quantile)}
\end{itemize}

%==============%
\bigskip
\textbf{Options}
\begin{itemize}
\item[(1)] 10904118 and 10575416
\item[(2)] 15259150 and 10575416
\item[(3)] 16776430 and 15390165
\item[(4)] 14191406  and 10904118
\end{itemize}

%-------------------------------------------------- %

\begin{framed}
	\begin{verbatim}
	https://d396qusza40orc.cloudfront.net/getdata%2Fjeff.jpg
	\end{verbatim}
\end{framed}


\begin{framed}
\begin{verbatim}
url <- "https://d396qusza40orc.cloudfront.net/getdata%2Fjeff.jpg"
file <- file.path(getwd(), "jeff.jpg")
download.file(url, file, mode = "wb", method = "curl")
img <- readJPEG(file, native = TRUE)
\end{verbatim}
\end{framed}
% > quantile(img, probs = c(0.3, 0.8))
% 30%       80% 
% -15259150 -10575416 
%------------------------------------------------------------------------------------------------------------%
\newpage
\subsection*{Question 3}
Load the Gross Domestic Product data for the 190 ranked countries in this data set: 

\begin{verbatim}
https://d396qusza40orc.cloudfront.net/getdata%2Fdata%2FGDP.csv 
\end{verbatim}
Load the educational data from this data set: 
\begin{verbatim}
https://d396qusza40orc.cloudfront.net/getdata%2Fdata%2FEDSTATS_Country.csv 
\end{verbatim}
Match the data based on the country shortcode. How many of the IDs match? Sort the data frame in descending order by GDP rank (so United States is last). 
\\
What is the 13th country in the resulting data frame? 
\bigskip
\textbf{Options}
\begin{itemize}
\item[1] 189, St. Kitts and Nevis
\item[2] 234, Spain
\item[3] 190, Spain
\item[4] 189, Spain
\item[5] 190, St. Kitts and Nevis
\item[6] 234, St. Kitts and Nevis
\end{itemize}

\textbf{Original data sources:} 
http://data.worldbank.org/data-catalog/GDP-ranking-table \\
http://data.worldbank.org/data-catalog/ed-stats\\



%189, St. Kitts and Nevis

\newpage
%------------------------------------------------------------------------------------------------------------%

Explanation
\begin{framed}
	\begin{verbatim}
	> url <- "https://d396qusza40orc.cloudfront.net/getdata%2Fdata%2FGDP.csv"
	> file <- file.path(getwd(), "GDP.csv")
	> download.file(url, file, method = "curl")
	> dtGDP <- data.table(read.csv(file, skip = 4, nrows = 215))
	> dtGDP <- dtGDP[X != ""]
	> dtGDP <- dtGDP[, list(X, X.1, X.3, X.4)]
	> setnames(dtGDP, c("X", "X.1", "X.3", "X.4"), c("CountryCode", "rankingGDP", "Long.Name", "gdp"))
	> url <- "https://d396qusza40orc.cloudfront.net/getdata%2Fdata%2FEDSTATS_Country.csv"
	> file <- file.path(getwd(), "EDSTATS_Country.csv")
	> download.file(url, file, method = "curl")
	> dtEd <- data.table(read.csv(file))
	> dt <- merge(dtGDP, dtEd, all = TRUE, by = c("CountryCode"))
	> sum(!is.na(unique(dt$rankingGDP)))
	[1] 189
	> dt[order(rankingGDP, decreasing = TRUE), list(CountryCode, Long.Name.x, Long.Name.y, rankingGDP, gdp)][13]
	CountryCode         Long.Name.x         Long.Name.y rankingGDP   gdp
	1:         KNA St. Kitts and Nevis St. Kitts and Nevis        178  767 
	\end{verbatim}
\end{framed}
%------------------------------------------------------ %
\newpage
%------------------------------------------------------------------------------------------------------------%
\subsection*{Question 4}
What is the average GDP ranking for the "High income: OECD" and "High income: nonOECD" group?
\textbf{Options}
\begin{itemize}
\item[(i)]  23, 45
\item[(ii)] 30, 37
\item[(iii)] 23, 30
\item[(iv)] 23.966667, 30.91304
\item[(v)] 32.96667, 91.91304
\item[(vi)] 133.72973, 32.96667
\end{itemize}

% 32.96667, 91.91304

%--------------------------------------------------- %
\begin{framed}
	\begin{verbatim}
	> dt[, mean(rankingGDP, na.rm = TRUE), by = Income.Group]
	Income.Group        V1
	1: High income: nonOECD  91.91304
	2:           Low income 133.72973
	3:  Lower middle income 107.70370
	4:  Upper middle income  92.13333
	5:    High income: OECD  32.96667
	6:                   NA 131.00000
	7:                            NaN
	\end{verbatim}
\end{framed}
%------------------------------------------------------------------------------------------------------------%
\newpage
\subsection*{Question 5}
Cut the GDP ranking into 5 separate quantile groups. Make a table versus Income.Group.
\\
 How many countries are Lower middle income but among the 38 nations with highest GDP?\\
\bigskip
\textbf{Options}
\begin{itemize}
\item[(1)] 0
\item[(2)]18
\item[(3)]13
\item[(4)] 5
\end{itemize}

%5

%--------------------------------------------------- %
\begin{framed}
	\begin{verbatim}
	> breaks <- quantile(dt$rankingGDP, 
	  	probs = seq(0, 1, 0.2), na.rm = TRUE)
	> dt$quantileGDP <- cut(dt$rankingGDP, breaks = breaks)
	> dt[Income.Group == "Lower middle income", .N, 
	   	by = c("Income.Group", "quantileGDP")]
	   	
	Income.Group quantileGDP  N
	1: Lower middle income (38.8,76.6] 13
	2: Lower middle income   (114,152]  8
	3: Lower middle income   (152,190] 16
	4: Lower middle income  (76.6,114] 12
	5: Lower middle income    (1,38.8]  5
	6: Lower middle income          NA  2
	\end{verbatim}
\end{framed}
\end{document}


Quiz 3
Attempts	Score
2/3	15/15
Question 1

The American Community Survey distributes downloadable data about United States communities. Download the 2006 microdata survey about housing for the state of Idaho using download.file() from here:

\begin{verbatim}
https://d396qusza40orc.cloudfront.net/getdata%2Fdata%2Fss06hid.csv
\end{verbatim}

and load the data into R. The code book, describing the variable names is here:
\begin{verbatim}
https://d396qusza40orc.cloudfront.net/getdata%2Fdata%2FPUMSDataDict06.pdf
\end{verbatim}

Create a logical vector that identifies the households on greater than 10 acres who sold more than \$10,000 worth of agriculture products. Assign that logical vector to the variable agricultureLogical. Apply the \texttt{which()} function like this to identify the rows of the data frame where the logical vector is TRUE. which(agricultureLogical) What are the first 3 values that result?

Answer

125, 238, 262

Explanation


