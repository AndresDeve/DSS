\documentclass[12pt]{article}
\usepackage{framed}
\usepackage{graphicx}
\begin{document}
	
\section*{The Data Scientist’s Toolbox - Week 2 Quiz}
	
	
	% Data Scientist Toolkit Week 2 Quiz
	
%------------------------------------------------------------------------------------ %
\subsection*{Question 1}
Which of the following commands will create a directory called data in your current working directory?
\begin{itemize}
\item[(i)] \texttt{mkdir /Users/data}
\item[(ii)] \texttt{mkdir data}
\item[(iii)] \texttt{cd /data}
\item[(iv)] \texttt{pwd data}
\end{itemize}
%-------------------------------------------------------------------------------------%
\subsection*{Question 2}
Which of the following will initiate a git repository locally?
\begin{itemize}
\item[(i)] \texttt{git init}
\item[(ii)] \texttt{git remote add}
\item[(iii)] \texttt{git boom}
\item[(iv)] \texttt{git pull}
\end{itemize}
%-------------------------------------------------------------------------------------%
\newpage
\subsection*{Question 3}
Suppose you have forked a repository called \textbf{\textit{datascientist}} on Github but it isn't on your local computer yet. Which of the following is the command to bring the directory to your local computer? (For this question assume that your user name is \texttt{username})

\begin{itemize}
\item \texttt{git init}
\item \texttt{git pull username https://github.com/username/datascientist.git}
\item \texttt{git clone https://github.com/username/datascientist.git}
\item \texttt{git pull https://github.com/username/datascientist.git}
\end{itemize}
%-------------------------------------------------------------------------------------%
\newpage

\subsection*{Question 4 - Markdown}
Which of the following will create a markdown document with a secondary heading saying "Data Science Specialization" and an unordered list with the following for bullet points: \textit{Uses R, Nine courses, Goes from raw data to data products}
\begin{itemize}
\item Option 1
\begin{framed}
\begin{verbatim}
*h2 Data Science Specialization 

* Uses R 
* Nine courses 
* Goes from raw data to data products
\end{verbatim}
\end{framed}
%-------------------------%
\item Option 2
\begin{framed}
\begin{verbatim}
*** Data Science Specialization 

* Uses R 
* Nine courses 
* Goes from raw data to data products
\end{verbatim}
\end{framed}
%--------------------------%
\item Option 3
\begin{framed}
\begin{verbatim}
## Data Science Specialization 

* Uses R 
* Nine courses 
* Goes from raw data to data products
\end{verbatim}
\end{framed}
\newpage
%--------------------------%
\item Option 4
\begin{framed}
\begin{verbatim}
### Data Science Specialization 

* Uses R 
* Nine courses 
* Goes from raw data to data products
\end{verbatim}
\end{framed}
%-----------------------%

\item Option 5
\begin{framed}
\begin{verbatim}
## Data Science Specialization 
li Uses R 
li Nine courses 
li Goes from raw data to data products
\end{verbatim}
\end{framed}

\item \textbf{Remark} \\
Try use each option when constructing a markdown file in github.
\end{itemize}


%-------------------------------------------------------------------------------------%
\newpage

\subsection*{Question 5}
Install and load the \textbf{\emph{KernSmooth}} \texttt{R} package. What does the copyright message say?

\begin{itemize}
\item[(i)] \texttt{Copyright KernSmooth 1997-2009}
\item[(ii)] \texttt{Copyright M. P. Wand}
\item[(iii)] \texttt{Copyright Coursera 2009-2013}
\item[(iv)] \texttt{Copyright M. P. Wand 1997-2009}
\end{itemize}
%-------------------------------------------------------------------------------------%
\end{document}
