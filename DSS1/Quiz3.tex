\subsection*{Question 1}
We take a random sample of individuals in a population and identify whether they smoke and if they have cancer. We observe that there is a strong relationship between whether a person in the sample smoked or not and whether they have lung cancer. 

We claim that the smoking is related to lung cancer in the larger population. We explain we think that the reason for this relationship is because cigarette smoke contains known carcinogens such as arsenic and benzene, which make cells in the lungs become cancerous.
\begin{itemize}
\item This is an example of an inferential data analysis.
\item This is an example of a mechanistic data analysis.
\item This is an example of an descriptive data analysis.
\item This is an example of a causal data analysis.
\end{itemize}
%------------------------------------------------------------------------------------------%
\subsection*{Question 2}
What is the most important thing in Data Science?
\begin{itemize}
\item Working with large data sets.
\item Hacking skills.
\item Knowing Hadoop and Pig.
\item The question you are trying to answer.
\end{itemize}
%------------------------------------------------------------------------------------------%
\subsection*{Question 3}
If the goal of a study was to relate Martha Stewart Living Subscribers to Our Site's Users based on the number of people that lived in each region of the US, what would be the potential problem? 



Link to image 
Source: http://xkcd.com/1138/
\begin{itemize}
\item There would be confounding because the number of people that live in an area is related to both Martha Stewart Living Subscribers and Our Site's Users.
\item We would be performing inference on the relationship between Martha Stewart Living Subscribers and Our Site's Users.
\item We wouldn't know the sensitivity of our predictions.
\item We wouldn't be able to estimate the variability in Martha Stewart Living Subscribers.
\end{itemize}
%------------------------------------------------------------------------------------------%
\subsection*{Question 4}
What is an experimental design tool that can be used to address variables that may be confounders at the design phase of an experiment?
\begin{itemize}
\item Randomization
\item Only using non-confounding variables.
\item Using data from a database.
\item Using all the data you have access too.
\end{itemize}
%------------------------------------------------------------------------------------------%
\subsection*{Question 5}
What is the reason behind the explosion of interest in big data?

\begin{itemize}
\item We recently discovered ways to use data to make predictions.
\item The price and difficulty of collecting and storing data has dramatically dropped.
\item We recently discovered ways to use data to answer scientific and business questions.
\item There have been massive improvements in machine learning algorithms.
\end{itemize}
%------------------------------------------------------------------------------------------%
\end{document}