setwd("~/Desktop/Online Coursera/Coursera-Exploratory-Data-Analysis/week3/")
## Hierarchical Clustering
# Agglomerative approach: find closest 2 things, put them together and 
#                         find next closest
# A defined distance, a merging approach
# Produce a tree
# Defination of Close: 
# Distance/similarity: continuous - euclidean distance; 
#                      binary - manhattan distance
set.seed(1234)
par(mar = c(0, 0, 0, 0))
x <- rnorm(12, mean = rep(1:3, each = 4), sd = 0.2)
y <- rnorm(12, mean = rep(c(1, 2, 1), each = 4), sd = 0.2)
plot(x, y, col = "blue", pch = 19, cex = 2)
text(x + 0.05, y + 0.05, labels = as.character(1:12))
# - dist
dataFrame <- data.frame(x = x, y = y)
distxy <- dist(dataFrame) # lower triangular matrix
hClustering <- hclust(distxy)
plot(hClustering)
# Prettier dendrograms
myplclust <- function(hclust, lab = hclust$labels, lab.col = rep(1, length(hclust$labels)), 
                      hang = 0.1, ...) {
    y <- rep(hclust$height, 2)
    x <- as.numeric(hclust$merge)
    y <- y[which(x < 0)]
    x <- x[which(x < 0)]
    x <- abs(x)
    y <- y[order(x)]
    x <- x[order(x)]
    plot(hclust, labels = F, hang = hang, ...)
    text(x = x, y = y[hclust$order] - (max(hclust$height)*hang), labels = lab[hclust$order], 
         col = lab.col[hclust$order], srt = 90, adj = c(1, 0.5), xpd = NA, ...)
}
myplclust(hClustering, lab = rep(1:3, each = 4), lab.col = rep(1:3, each = 4))
# heatmap()
set.seed(143)
dataMatrix <- as.matrix(dataFrame)[sample(1:12), ]
heatmap(dataMatrix)
# Unstable when changing a few points, having missing values or changing scales of data
# It is deterministic

# K-means Clustering
# define close, group things, visulize and interpret grouping
# A partioning approach: fix a number of clusters, get "centroids" of each cluster,
# assign things to closest centroid, recalculate centroids
# A defined distance metric, a number of clusters, an initial guess as to cluster centroids
# Produce the final estimate of cluster centroids and an assignment of each point to clusters
set.seed(1234)
par(mar = c(0, 0, 0, 0))
x <- rnorm(12, mean = rep(1:3, each = 4), sd = 0.2)
y <- rnorm(12, mean = rep(c(1, 2, 1), each = 4), sd = 0.2)
plot(x, y, col = "blue", pch = 19, cex = 2)
text(x + 0.05, y + 0.05, labels = as.character(1:12))

# kmeans()
dataFrame <- data.frame(x, y)
kmeansObj <- kmeans(dataFrame, centers = 3)
names(kmeansObj)
kmeansObj$cluster

par(mar = rep(0.2, 4))
plot(x, y, col = kmeansObj$cluster, pch = 19, cex = 2)
points(kmeansObj$centers, col = 1:3, pch = 3, cex = 3, lwd = 3)

# Heatmaps
set.seed(1234)
dataMatrix <- as.matrix(dataFrame)[sample(1:12), ]
kmeansObj2 <- kmeans(dataMatrix, centers = 3)
par(mfrow = c(1, 2), mar = c(2, 4, 0.1, 0.1))
image(t(dataMatrix)[, nrow(dataMatrix):1], yaxt = "n")
image(t(dataMatrix)[, order(kmeansObj$cluster)], yaxt = "n")
# K-means is not deterministic: different # of cluesters/iterations

## Dimension Reduction: PCA & SVD
# Matrix data
set.seed(12345)
par(mar = rep(0.2, 4))
dataMatrix <- matrix(rnorm(400), nrow = 40)
image(1:10, 1:40, t(dataMatrix)[, nrow(dataMatrix):1])
par(mar = rep(0.2, 4))
heatmap(dataMatrix)
# Add a pattern
set.seed(678910)
for (i in 1:40) {
    # flip a coin
    coinFlip <- rbinom(1, size = 1, prob = 0.5)
    if(coinFlip) {
        dataMatrix[i, ] <- dataMatrix[i, ] + rep(c(0, 3), each = 5)
    }
}
par(mar = rep(0.2, 4))
image(1:10, 1:40, t(dataMatrix)[, nrow(dataMatrix):1])
heatmap(dataMatrix)

# Patterns in rows and columns
hh <- hclust(dist(dataMatrix))
dataMatrixOrdered <- dataMatrix[hh$order, ]
par(mfrow = c(1, 3))
image(t(dataMatrixOrdered)[, nrow(dataMatrixOrdered):1])
plot(rowMeans(dataMatrixOrdered), 40:1, , xlab = "Row Mean", ylab = "Row", pch = 19)
plot(colMeans(dataMatrixOrdered), xlab = "Column", ylab = "Column Mean", pch = 19)

## Components of the SVD
svd1 <- svd(scale(dataMatrixOrdered))
par(mfrow = c(1, 3))
image(t(dataMatrixOrdered)[, nrow(dataMatrixOrdered):1])
plot(svd1$u[, 1], 40:1, , xlab = "Row", ylab = "First left singular vector", 
     pch = 19)
plot(svd1$v[, 1], xlab = "Column", ylab = "First right singular vector", pch = 19)

# - variance explained
par(mfrow = c(1, 2))
plot(svd1$d, xlab = "Column", ylab = "Singular value", pch = 19)
plot(svd1$d^2 / sum(svd1$d^2), xlab = "Column", ylab = "Prop. of variance explained", 
     pch = 19)

# relationship to principal components
svd1 <- svd(scale(dataMatrixOrdered))
pca1 <- prcomp(dataMatrixOrdered, scale = TRUE)
plot(pca1$rotation[, 1], svd1$v[, 1], pch = 19, xlab = "Principal Component 1", 
     ylab = "Right Singular Vector 1")
abline(c(0, 1))

# Components of the svd - variance explained
constantMatrix <- dataMatrixOrdered*0
for(i in 1:dim(dataMatrixOrdered)[1]){constantMatrix[i,] <- rep(c(0,1),each=5)}
svd1 <- svd(constantMatrix)
par(mfrow = c(1,3))
image(t(constantMatrix)[, nrow(constantMatrix):1])
plot(svd1$d, xlab = "Column", ylab="Singular value", pch = 19)
plot(svd1$d^2 / sum(svd1$d^2), xlab = "Column", 
     ylab = "Prop. of variance explained", pch = 19)

# Add a second pattern
set.seed(678910)
for (i in 1:40) {
    # flip a coin
    coinFlip1 <- rbinom(1, size = 1, prob = 0.5)
    coinFlip2 <- rbinom(1, size = 1, prob = 0.5)
    # if coin is heads add a common pattern to that row
    if (coinFlip1) {
        dataMatrix[i, ] <- dataMatrix[i, ] + rep(c(0, 5), each = 5)
    }
    if (coinFlip2) {
        dataMatrix[i, ] <- dataMatrix[i, ] + rep(c(0, 5), 5)
    }
}
hh <- hclust(dist(dataMatrix))
dataMatrixOrdered <- dataMatrix[hh$order, ]

# SVD - true patterns
svd2 <- svd(scale(dataMatrixOrdered))
par(mfrow = c(1, 3))
image(t(dataMatrixOrdered)[, nrow(dataMatrixOrdered):1])
plot(rep(c(0, 1), each = 5), pch = 19, xlab = "Column", ylab = "Pattern 1")
plot(rep(c(0, 1), 5), pch = 19, xlab = "Column", ylab = "Pattern 2")

# v and patterns of variance in rows
svd2 <- svd(scale(dataMatrixOrdered))
par(mfrow = c(1, 3))
image(t(dataMatrixOrdered)[, nrow(dataMatrixOrdered):1])
plot(svd2$v[, 1], pch = 19, xlab = "Column", ylab = "First right singular vector")
plot(svd2$v[, 2], pch = 19, xlab = "Column", ylab = "Second right singular vector")

# d and variance explained
svd1 <- svd(scale(dataMatrixOrdered))
par(mfrow = c(1, 2))
plot(svd1$d, xlab = "Column", ylab = "Singular value", pch = 19)
plot(svd1$d^2/sum(svd1$d^2), xlab = "Column", ylab = "Percent of variance explained", 
     pch = 19)

## Missing values
dataMatrix2 <- dataMatrixOrdered
# Randomly insert some missing data
dataMatrix2[sample(1:100, size = 40, replace = FALSE)] <- NA
svd1 <- svd(scale(dataMatrix2))  ## Doesn't work!

source("http://bioconductor.org/biocLite.R")
biocLite("impute")
library(impute)  ## Available from http://bioconductor.org
dataMatrix2 <- dataMatrixOrdered
dataMatrix2[sample(1:100, size = 40, replace = FALSE)] <- NA
dataMatrix2 <- impute.knn(dataMatrix2)$data
svd1 <- svd(scale(dataMatrixOrdered))
svd2 <- svd(scale(dataMatrix2))
par(mfrow = c(1,2))
plot(svd1$v[,1], pch = 19)
plot(svd2$v[,1], pch = 19)

# Face example
load("data/face.rda")
image(t(faceData)[, nrow(faceData):1])

# Variance explained
svd1 <- svd(scale(faceData))
plot(svd1$d^2 / sum(svd1$d^2), pch = 19, 
     xlab = "Singular vector", ylab = "Variance explained")

# Create approximatation
vd1 <- svd(scale(faceData))
## Note that %*% is matrix multiplication

# Here svd1$d[1] is a constant
approx1 <- svd1$u[, 1] %*% t(svd1$v[, 1]) * svd1$d[1]

# In these examples we need to make the diagonal matrix out of d
approx5 <- svd1$u[, 1:5] %*% diag(svd1$d[1:5]) %*% t(svd1$v[, 1:5])
approx10 <- svd1$u[, 1:10] %*% diag(svd1$d[1:10]) %*% t(svd1$v[, 1:10])

# Plot approximations
par(mfrow = c(1, 4))
image(t(approx1)[, nrow(approx1):1], main = "(a)")
image(t(approx5)[, nrow(approx5):1], main = "(b)")
image(t(approx10)[, nrow(approx10):1], main = "(c)")
image(t(faceData)[, nrow(faceData):1], main = "(d)")  ## Original data

# Scale matters, PC's/SV/s may mix real patterns

## Work with Color in R plots
# 
heat.colors()
topo.colors()
# grDevices
# colorRamp(takes 0-1), colorRampPalette
pal <- colorRamp(c("red", "blue"))
pal(0)
pal(1)
pal(0.5)
pal(seq(0, 1, len = 10))
pal <- colorRampPalette(c("red", "yellow"))
pal(2)
pal(10)
# 3 types of palettes: sequential(low to high), 
#                      diverging, qualitative
library(RColorBrewer)
cols <- brewer.pal(3, "BuGn")
cols
pal <- colorRampPalette(cols)
image(volcano, col = pal(20)) # 20 colors
# The smoothScatter
x <- rnorm(10000)
y <- rnorm(10000)
smoothScatter(x, y)

## rgb function, transparency - alpha parameter
library(colorspace)
x <- rnorm(1000)
y <- rnorm(1000)
plot(x, y, col = rgb(0, 0, 0, 0.2), pch = 19)
Status API Training
