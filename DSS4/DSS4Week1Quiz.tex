\documentclass[french]{article}
\usepackage[utf8]{inputenc}
\usepackage[T1]{fontenc}
\usepackage{lmodern}
\usepackage[a4paper]{geometry}
\usepackage{babel}
\begin{document}

\section*{Data Specialization Module 7 - Quiz - Week 1}
\subsection*{Question 1}

%---------------------------------------- %
\newpage
\subsection*{Question 2}


%---------------------------------------- %
\newpage
\subsection*{Question 3}

\end{document}

%------------------------------------------------------------------------------------------%
Quiz 1
Attempts	Score
2/3	10/10
Question 1

Which of the following is a principle of analytic graphics?

Answer

Show comparisons

Question 2

What is the role of exploratory graphs in data analysis?

Answer

They are typically made very quickly.

Question 3

Which of the following is true about the base plotting system?

Answer

Plots are created and annotated with separate functions

Explanation

Functions like 'plot' or 'hist' typically create the plot on the graphics device and functions like 'lines', 'text', or 'points' will annotate or add data to the plot.

Question 4

Which of the following is an example of a valid graphics device in R?

Answer

A PDF file

Question 5

Which of the following is an example of a vector graphics device in R?

Answer

SVG

Question 6

Bitmapped file formats can be most useful for

Answer

Scatterplots with many many points

Question 7

Which of the following functions is typically used to add elements to a plot in the base graphics system?

Answer

lines()

Question 8

Which function opens the screen graphics device on Windows?

Answer

windows()

Question 9

What does the 'pch' option to par() control?

Answer

the plotting symbol/character in the base graphics system

Question 10

If I want to save a plot to a PDF file, which of the following is a correct way of doing that?

Answer

Construct the plot on the screen device and then copy it to a PDF file with dev.copy2pdf()
